
\chapter{Molcas}

\section{Introduction}

OpenMolcas is a powerful quantum chemistry package particularly suited for multireference methods like CASSCF and CASPT2. It supports a wide range of post-HF methods and spin-orbit coupling calculations.

\section{Active Space Selection}

Use the Grid Viewer (GV) to inspect molecular orbitals and guide the choice of active space:

\begin{itemize}
  \item Choose $\pi$ and $\pi^*$ orbitals for conjugated systems
  \item Include lone pairs or nonbonding orbitals when relevant
  \item Exclude very low or very high virtual orbitals unless involved in excitation
\end{itemize}

\section{Input Structure}

Molcas input files typically include modules:

\begin{verbatim}
&GATEWAY
Title = molecule
Coord = molecule.xyz
Basis = cc-pVDZ

&SEWARD

&SCF

&RASSCF
Inactive = 10
Active = 4
Ras2 = 4
Nactel = 4 0 0

&CASPT2

&RASSI
NrofJobI = 2
\end{verbatim}

\section{Optimization and State Averaging}

Use \texttt{SLAPAF} for geometry optimization.

State averaging:

\begin{verbatim}
CIROOT = [2,2,1]
\end{verbatim}

Distribute weights across singlet/triplet states appropriately.

\section{Spectroscopic Applications}

Use \texttt{RASSI} for transition dipoles, SOC, and transition intensities.

\begin{itemize}
  \item Combine ground and excited states from RASSCF/CASPT2
  \item Calculate absorption and emission spectra (ESA/GSA)
\end{itemize}

\section{Best Practices}

\begin{itemize}
  \item Use Cholesky decomposition to speed up integrals
  \item Start with small active spaces, then increase
  \item Always check orbitals and CI vectors visually
  \item Use \texttt{\&ALASKA} for analytical gradients where possible
\end{itemize}
