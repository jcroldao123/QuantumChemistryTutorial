\chapter{ORCA}

\section{Introduction}

ORCA is a flexible, efficient, and easy-to-use general-purpose quantum chemistry package that is particularly strong in spectroscopy and open-shell systems.

\section{Input File Structure}

ORCA input files have the extension \texttt{.inp} and typically include:

\begin{itemize}
    \item Calculation keywords
    \item Molecular structure (XYZ format)
    \item Basis set and method
\end{itemize}

\subsection*{Example ORCA Input}

\begin{verbatim}
! B3LYP 6-31G* Opt Freq

* xyz 0 1
O      0.000000    0.000000    0.000000
H      0.757160    0.586260    0.000000
H     -0.757160    0.586260    0.000000
*
\end{verbatim}

\section{Running a Calculation}

Use the following command to run ORCA:

\begin{verbatim}
orca input.inp > output.out
\end{verbatim}

It will generate:
\begin{itemize}
    \item \texttt{.out} – main output
    \item \texttt{.gbw} – binary wavefunction
    \item \texttt{.hess}, \texttt{.engrad} – gradient and Hessian if requested
\end{itemize}

\section{TDDFT and Excited States}

To run a TDDFT calculation:

\begin{verbatim}
! B3LYP 6-31G* TDDFT TightSCF
\end{verbatim}

You can also request specific excited states:

\begin{verbatim}
%tddft
   nroots 10
   maxdim 50
end
\end{verbatim}

\section{Solvent Effects}

Use the CPCM or SMD model:

\begin{verbatim}
%cpcm
  smd true
  solvent "Water"
end
\end{verbatim}

\section{Visualizing Results}

Use Avogadro, Chemcraft, or Gabedit to visualize the geometry, MOs, and vibrations.

For spectra, ORCA can output to `.csv` and `.dat` formats, readable by plotting tools.

\section*{Notes}
\begin{itemize}
    \item ORCA input is sensitive to formatting—avoid tabs and use correct case.
    \item The \texttt{.gbw} file can be reused for restart or post-processing.
    \item ORCA can run in parallel via \texttt{orca input.inp > output.out &}
\end{itemize}


\section*{Advanced ORCA Usage}

\subsection*{TDDFT with Spin-Orbit Coupling}

To include spin-orbit coupling (SOC) in TDDFT calculations:

\begin{verbatim}
! B3LYP def2-SVP TightSCF TDDFT SOC
%tddft
  nroots 10
end
\end{verbatim}

\subsection*{Solvent Models}

ORCA supports implicit solvation via CPCM or SMD:

\begin{verbatim}
%cpcm
  smd true
  solvent "Acetonitrile"
end
\end{verbatim}

\subsection*{Using Chemcraft and Multiwfn}

\begin{itemize}
  \item Visualize molecular orbitals and vibrational modes with Chemcraft.
  \item Analyze electron density difference with Multiwfn from ORCA cube files.
\end{itemize}

\subsection*{Excited-State Analysis}

Check the \texttt{.out} file for:

\begin{itemize}
  \item Excitation energies and oscillator strengths
  \item Orbital transitions (e.g., HOMO → LUMO)
  \item Assignment of charge transfer (CT) or local excitation (LE)
\end{itemize}

\subsection*{Tips}

\begin{itemize}
  \item Use \texttt{maxdim} and \texttt{tol} in TDDFT block to improve convergence.
  \item Always verify orbital character with visual inspection.
  \item Use \texttt{NumFreq} for numerical frequencies when Hessians fail.
\end{itemize}
