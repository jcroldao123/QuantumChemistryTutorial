\chapter{ORCA}

\section{Introduction}

ORCA is a flexible and powerful quantum chemistry package designed for modern electronic structure methods, including DFT, TDDFT, and multireference methods.

\section{TDDFT Calculations}

To run TDDFT with ORCA:

\begin{verbatim}
! B3LYP def2-SVP TightSCF TDDFT
%tddft
  nroots 10
end
\end{verbatim}

\section{Spin-Orbit Coupling}

To include spin-orbit coupling (SOC):

\begin{verbatim}
! B3LYP def2-TZVP SOC TDDFT
\end{verbatim}

\section{Solvation Models}

You can use CPCM or SMD for solvation:

\begin{verbatim}
%cpcm
  smd true
  solvent "Acetonitrile"
end
\end{verbatim}

\section{Visualization and Analysis}

\begin{itemize}
  \item Use \texttt{orca_plot} to extract orbital plots.
  \item Use Chemcraft or Avogadro to visualize orbitals and cube files.
  \item Use Multiwfn to analyze electron density and ESP maps.
\end{itemize}

\section{Excited-State Analysis}

Check the \texttt{.out} file for:

\begin{itemize}
  \item Excitation energies and oscillator strengths
  \item Orbital contributions: e.g., HOMO $\rightarrow$ LUMO
  \item Natural transition orbitals (NTOs)
\end{itemize}

\section{Tips}

\begin{itemize}
  \item Use \texttt{maxdim} and \texttt{tol} in \texttt{\%tddft} block for convergence tuning.
  \item Inspect orbital contributions for charge transfer vs local excitations.
  \item Use \texttt{NumFreq} for numerical frequencies if analytic gradients are not available.
\end{itemize}