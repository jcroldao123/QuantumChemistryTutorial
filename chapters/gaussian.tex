\chapter{Gaussian}

\section{Molecule Construction}

\subsection*{General Ideas}
Use GaussView or similar tools to construct the initial geometry. Proper naming of files is encouraged:

\begin{verbatim}
Molecule_Symmetry_State_Functional_BasisSet_CalcType_Solvent.gjf
\end{verbatim}

\textbf{Example:}

\begin{verbatim}
DSB_C2H_GS_BL_6311gd_Opt-Freq_Vac.gjf
\end{verbatim}

\begin{itemize}
\item \textbf{State}: GS, S1, T1, etc.
\item \textbf{Functional}: B3LYP, CAM-B3LYP, M06HF, etc.
\item \textbf{BasisSet}: 6-31G(d), 6-311G(d,p)
\item \textbf{CalcType}: Opt, Freq, SP, TDDFT, etc.
\end{itemize}

\subsection*{Tips for Optimization Verification}
\begin{itemize}
\item Use GaussView: \textit{Read intermediate geometries (optimization)}
\item For excited state geometry optimization: use the \textbf{penultimate} geometry when resubmitting.
\end{itemize}

\section{Optimization and Frequency Calculation}

\subsection*{Input File Preparation}
Use keywords such as:

\begin{verbatim}
#p B3LYP/6-311G(d) Opt Freq
\end{verbatim}

\subsection*{Job Submission}
Submit your \texttt{.gjf} files to the cluster with \texttt{.pbs} or SLURM scripts.

Example PBS script:

\begin{verbatim}
#!/bin/bash
#PBS -N my_gaussian_job
#PBS -l nodes=1:ppn=8
#PBS -l walltime=12:00:00
#PBS -j oe
cd $PBS_O_WORKDIR
g16 < input.gjf > output.log
\end{verbatim}

\subsection*{Output File Analysis}
Use GaussView or analyze manually the \texttt{.log} and \texttt{.chk} files.

\section{File Conversion and Orbital Analysis}

\subsection*{Generate .fchk Files}
Use \texttt{formchk}:

\begin{verbatim}
formchk input.chk output.fchk
\end{verbatim}

\subsection*{Molecular Orbitals}
Load \texttt{.fchk} in GaussView for orbital visualization. Use isovalue settings to adjust surface display.

\section{Linear Response (TDDFT) Calculations}

\subsection*{Input File}
Example:

\begin{verbatim}
#p CAM-B3LYP/6-311G(d,p) TD(NStates=10)
\end{verbatim}

\subsection*{Analysis}
Examine the excited states section in the \texttt{.log} file. Extract oscillator strengths, transition energies, and orbital contributions.

\section{Complex Systems and Spectroscopy}

\subsection*{Metal Complex Optimization}
Use Effective Core Potentials (ECP) for heavy atoms.

Example basis set line:

\begin{verbatim}
LANL2DZ
\end{verbatim}

\subsection*{Raman Calculations}
Include the \texttt{Freq=Raman} keyword and analyze intensity data in the output.

\section*{Notes}
\begin{itemize}
\item Intermediate \texttt{.log} files can be used to visualize optimization progress.
\item Never use the final geometry of a failed optimization.
\item Always verify imaginary frequencies in frequency calculations.
\end{itemize}


\section*{Advanced Topics in Gaussian}

\subsection*{Generating Additional Files}

\begin{itemize}
  \item Use \texttt{formchk} to convert \texttt{.chk} to \texttt{.fchk}.
  \item Use \texttt{cubegen} to generate cube files for densities or MOs.
  \item Use \texttt{pop=full} or \texttt{pop=chelpg} to extract atomic charges.
\end{itemize}

Example:

\begin{verbatim}
formchk molecule.chk molecule.fchk
cubegen 0 density=SCF molecule.fchk density.cube -2 h
\end{verbatim}

\subsection*{Plotting Electrostatic Potential}

\begin{itemize}
  \item Use GaussView to open \texttt{.fchk} and generate electrostatic potential maps.
  \item Export 2D or 3D plots as \texttt{.eps} for publications.
\end{itemize}

\subsection*{Functional Tuning (ω tuning)}

\begin{itemize}
  \item Use CAM-B3LYP and related functionals for charge-transfer states.
  \item Optimize the $\omega$ parameter using ionization potential/electron affinity matching.
\end{itemize}

\subsection*{TDA and TDDFT}

\begin{itemize}
  \item TDDFT: \texttt{\# TD(NStates=10) CAM-B3LYP/6-31G(d)}
  \item TDA: Add keyword \texttt{TD=(NStates=10,TDA)} for improved stability
\end{itemize}

\subsection*{Analyzing States and Orbitals}

\begin{itemize}
  \item Use GaussView to inspect HOMO, LUMO and transitions.
  \item Use Multiwfn for quantitative analysis of transition density matrix.
  \item Verify transition character (n→π*, π→π*, CT) using orbital visualization.
\end{itemize}

\subsection*{Population Analysis}

\begin{itemize}
  \item Hirshfeld: use \texttt{pop=hirshfeld}
  \item CM5 charges: use \texttt{pop=cm5}
\end{itemize}

\subsection*{Tips}

\begin{itemize}
  \item Do not analyze failed optimizations.
  \item Always visualize excited states with caution.
  \item Compare different methods: B3LYP, CAM-B3LYP, M06HF.
\end{itemize}
