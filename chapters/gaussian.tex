\chapter{Gaussian}

\section{Introduction}

Gaussian is a widely used quantum chemistry program that supports a range of methods including HF, DFT, MP2, and TDDFT.

\section{Basic Input}

\begin{verbatim}
#P B3LYP/6-31G(d) Opt Freq

Title Card Required

0 1
C  0.000  0.000  0.000
H  0.000  0.000  1.089
...
\end{verbatim}

\section{Functional Tuning ($\omega$ tuning)}

\subsection*{Generating Additional Files}

\begin{itemize}
  \item Use \texttt{formchk} to convert \texttt{.chk} to \texttt{.fchk}.
  \item Use \texttt{cubegen} to generate cube files for densities or MOs.
  \item Use \texttt{pop=full} or \texttt{pop=chelpg} to extract atomic charges.
\end{itemize}

Example:

\begin{verbatim}
formchk molecule.chk molecule.fchk
cubegen 0 density=SCF molecule.fchk density.cube -2 h
\end{verbatim}

\subsection*{Plotting Electrostatic Potential}

\begin{itemize}
  \item Use GaussView to open \texttt{.fchk} and generate electrostatic potential maps.
  \item Export 2D or 3D plots as \texttt{.eps} for publications.
\end{itemize}

\subsection*{Functional Tuning ($\omega$ tuning)}

\begin{itemize}
  \item Use CAM-B3LYP and related functionals for charge-transfer states.
  \item Optimize the $\omega$ parameter using ionization potential/electron affinity matching.
\end{itemize}