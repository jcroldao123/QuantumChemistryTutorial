\chapter{Dalton}

\section{Introduction}

Dalton is a powerful electronic structure program for calculating molecular properties such as polarizabilities, electronic excitation energies, and multipole moments.

\section{Input Files}

A Dalton calculation requires at least two files:
\begin{itemize}
    \item \texttt{.mol} – molecular structure and geometry
    \item \texttt{.dal} – specification of the calculation
\end{itemize}

\subsection*{Example \texttt{.mol} file}

\begin{verbatim}
Charge=0  Multiplicity=1
O     0.000000    0.000000    0.000000
H     0.757160    0.586260    0.000000
H    -0.757160    0.586260    0.000000
\end{verbatim}

\subsection*{Example \texttt{.dal} file}

\begin{verbatim}
**DALTON INPUT
.RUN WAVE FUNCTIONS
**WAVE FUNCTIONS
.HF
**END OF INPUT
\end{verbatim}

\section{Running the Calculation}

You can execute the calculation using:

\begin{verbatim}
dalton input -mol input.mol -dal input.dal
\end{verbatim}

Adjust this command as necessary for SLURM or PBS batch submission.

\section{Output Analysis}

Dalton produces a \texttt{.out} file which contains the results of the calculation. Key elements include:

\begin{itemize}
    \item Total energy
    \item Final geometry
    \item MO coefficients
    \item Transition properties
\end{itemize}

Examine the output with terminal tools like \texttt{less}, \texttt{grep} or use a text editor.

\section*{Notes}
\begin{itemize}
    \item Dalton requires properly formatted input files to function.
    \item Errors are usually caught during the input parsing phase.
\end{itemize}


\section*{Advanced Usage and TDDFT in Dalton}

\subsection*{TDDFT Setup}

To run TDDFT in Dalton, your \texttt{.dal} file should include:

\begin{verbatim}
**DALTON INPUT
.RUN RESPONSE
**RESPONS
.TDDFT
*READ
NSTATE=10
**END OF INPUT
\end{verbatim}

\subsection*{Spin-Orbit Coupling (SOC)}

For SOC calculations, Dalton requires proper symmetry and relativistic settings:

\begin{verbatim}
**HAMILTONIAN
.SPIN-ORBIT
\end{verbatim}

\subsection*{Reusing Occupation Information}

You may reuse converged occupation from one run to initialize another. Use \texttt{.mol} with previous orbital information or rename generated files accordingly.

\subsection*{Solvent Effects}

Use PCM for implicit solvent:

\begin{verbatim}
**PCM
.SOLVENT
WATER
\end{verbatim}

\subsection*{Transition Properties and Analysis}

Dalton can compute transition dipole moments, oscillator strengths, and excitation energies. These appear in the \texttt{.out} file and can be parsed with scripts.

\subsection*{Tips for Calculations with Metal Complexes}

\begin{itemize}
  \item Use effective core potentials (ECPs) if available.
  \item Choose suitable basis sets (e.g., def2-TZVP for metals).
  \item Symmetry can affect convergence and property calculations — disable if needed.
\end{itemize}
