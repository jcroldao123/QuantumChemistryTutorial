\chapter{FCclasses}

\section{Introduction}

FCclasses is a tool for simulating vibronic spectra using the Franck-Condon (FC) approximation and Duschinsky rotation. It allows the modeling of absorption and emission spectra including vibrational structure.

\section{Input Preparation}

FCclasses requires:

\begin{itemize}
    \item Normal mode frequencies and displacement vectors for ground and excited states
    \item Harmonic force constants
    \item Duschinsky rotation matrix
\end{itemize}

These can be obtained from quantum chemistry software (e.g., Gaussian) and post-processed.

\section{Key Files and Format}

Main input files:

\begin{itemize}
    \item \texttt{freq.g}: frequencies and normal modes of ground state
    \item \texttt{freq.e}: frequencies and normal modes of excited state
    \item \texttt{geom.g}, \texttt{geom.e}: equilibrium geometries
    \item \texttt{fcclasses.inp}: parameter file
\end{itemize}

\subsection*{Sample \texttt{fcclasses.inp}}

\begin{verbatim}
&input
  temp = 298.0,
  e0 = 0.0,
  broad = 200.0,
  maxquanta = 10,
  intensity = 1,
  npoints = 1000,
  xmin = 0.0,
  xmax = 5.0
/
\end{verbatim}

\section{Running FCclasses}

Run the program using:

\begin{verbatim}
fcclasses < fcclasses.inp > fcclasses.out
\end{verbatim}

Outputs include:

\begin{itemize}
    \item \texttt{fcclasses.out} – summary of input and computational results
    \item \texttt{absorption.dat}, \texttt{emission.dat} – spectra
    \item \texttt{stick.dat} – discrete transitions
\end{itemize}

\section{Analysis and Plotting}

Spectra can be visualized using plotting software (e.g., gnuplot, matplotlib):

\begin{verbatim}
plot "absorption.dat" using 1:2 with lines
\end{verbatim}

\section*{Notes}

\begin{itemize}
    \item Frequencies must be in the same units and consistent across files.
    \item Geometry alignment between states is critical.
    \item Scripts can automate file extraction from Gaussian outputs.
\end{itemize}


\section*{Advanced Use Cases for FCclasses}

\subsection*{Simplified Workflow}

FCclasses can work with simplified files from Gaussian:

\begin{itemize}
  \item Convert log files to \texttt{.fchk} using \texttt{formchk}
  \item Extract normal modes and displacements using custom Python scripts or Multiwfn
\end{itemize}

\subsection*{Absorption and Emission Spectra}

You can simulate both:

\begin{itemize}
  \item AH (Adiabatic Hessian) approach – uses optimized geometries for each state
  \item TI (Time-Independent) approximation – using a single geometry
\end{itemize}

\subsection*{Dark States and Intensity}

Some excited states may be dark (oscillator strength $\approx 0$), but FCclasses can still simulate vibrational transitions for these states. Adjust broadening or restrict modes to better match experimental shapes.

\subsection*{Troubleshooting Tips}

\begin{itemize}
  \item Use consistent basis sets and functionals across states
  \item Check Duschinsky rotation matrix dimensions and sanity
  \item Align geometries before computing displacements
\end{itemize}

\subsection*{Automated Input Preparation}

Scripts can assist in generating:

\begin{itemize}
  \item Geometry files from Gaussian logs
  \item Displacement vectors
  \item FCclasses input files (\texttt{fcclasses.inp})
\end{itemize}

\subsection*{Visualization}

Plot stick spectra and convoluted spectra for absorption/emission using:

\begin{verbatim}
gnuplot
matplotlib
xmgrace
\end{verbatim}
