\chapter{Turbomole}

\section{Introduction}

Turbomole is an efficient quantum chemistry software package for performing Hartree-Fock, DFT, MP2 and related calculations.

It is driven primarily through terminal-based scripts and text files.

\section{Setting Up a Calculation}

To initialize a working directory, use the \texttt{define} script:

\begin{verbatim}
define
\end{verbatim}

You will be prompted to input:
\begin{itemize}
    \item Molecular geometry (cartesian or internal coordinates)
    \item Basis sets and atomic types
    \item Charge and multiplicity
    \item SCF and calculation method
\end{itemize}

After setup, several files are generated including:
\begin{itemize}
    \item \texttt{control} – main input file
    \item \texttt{coord} – geometry
    \item \texttt{basis} – basis set definitions
\end{itemize}

\section{Running Jobs}

Use specific modules for different job types. For example:

\begin{verbatim}
dscf
ridft
mpgrad
escf
\end{verbatim}

These programs read the \texttt{control} file and produce output in standard text files.

\section{Analyzing Output}

After execution, key results are stored in:

\begin{itemize}
    \item \texttt{job.last} – final log
    \item \texttt{energy} – total energy summary
    \item \texttt{gradient} – forces
    \item \texttt{mos} – molecular orbitals
\end{itemize}

You can visualize orbitals with external tools like \texttt{Turbomole-TurboPlot} or export for other viewers.

\section{Sample Workflow}

\begin{verbatim}
define
dscf > dscf.out
ridft > ridft.out
\end{verbatim}

Check each step and review \texttt{control} parameters as needed.

\section*{Notes}
\begin{itemize}
    \item Use \texttt{t2x} to convert Turbomole output to other formats.
    \item All files are text-based and editable.
    \item Scripts like \texttt{jobex} automate geometry optimization loops.
\end{itemize}


\section*{Advanced Tips for Turbomole}

\subsection*{Input File Setup}

Use the \texttt{define} script interactively to prepare input files:

\begin{itemize}
  \item Geometry: read from \texttt{coord} or manually input
  \item Charge and multiplicity: set in \texttt{control}
  \item Basis set: select from internal list (e.g., def-SVP, def2-TZVP)
\end{itemize}

\subsection*{Geometry Conversion}

Convert coordinates from Gaussian (.gjf) or ORCA (.xyz) using external tools or scripts. Make sure the atom ordering and units are correct.

\subsection*{Vertical Transitions with ESCF}

Add to the \texttt{control} file:

\begin{verbatim}
$escf
  irrep=a1
  nroots=10
\end{verbatim}

Then run:

\begin{verbatim}
escf > escf.out
\end{verbatim}

\subsection*{Job Management}

Use \texttt{jobex} for geometry optimization:

\begin{verbatim}
jobex -gcart > opt.out
\end{verbatim}

Use \texttt{ridft}, \texttt{dscf} and \texttt{grad} for specific tasks.

\subsection*{Output Files and Analysis}

Key outputs:

\begin{itemize}
  \item \texttt{control} – main configuration
  \item \texttt{energy} – SCF energy values
  \item \texttt{mos} – molecular orbitals
  \item \texttt{gradient} – for optimization steps
\end{itemize}

\subsection*{Visualization and Conversion}

Use \texttt{t2x} and tools like Avogadro or VMD for orbital and structure visualization. You may convert Turbomole output to cube or Molden formats.


\section{Getting Started}

\subsection*{TmoleX GUI}

TmoleX is a graphical interface for Turbomole that allows easier creation and submission of jobs. It is available for Linux and Windows and is useful for geometry visualization, setup, and basic analysis.

\subsection*{Shell Setup}

To load Turbomole:

\begin{verbatim}
source /opt/turbomole/Config_turbo_env
\end{verbatim}

\section{Ground-State Optimization}

Use the following commands in the working directory:

\begin{verbatim}
define
jobex -ri -c 200
\end{verbatim}

After geometry optimization:

\begin{verbatim}
aoforce > freq.out
\end{verbatim}

To extract frequencies:

\begin{verbatim}
grep -A1 'frequency' freq.out
\end{verbatim}

\section{Excited-State Calculations}

Perform TDDFT vertical excitation calculations:

\begin{verbatim}
escf > escf.out
\end{verbatim}

For excited-state optimizations:

\begin{verbatim}
jobex -ri -level escf -c 200
\end{verbatim}

Calculate frequencies in excited states:

\begin{verbatim}
escf > escf.out
aoforce -level escf > freq_es.out
\end{verbatim}

\section{UV-Vis Spectrum Generation}

Use the \texttt{t2x} and \texttt{spectrum} tools or post-process with Multiwfn.

Plot oscillator strengths vs excitation energy using:

\begin{verbatim}
grep 'excitation energy' escf.out
\end{verbatim}

\section{Troubleshooting and Cluster Execution}

\begin{itemize}
  \item Fix permission issues with:
  \begin{verbatim}
  chmod -R u+rwX .
  \end{verbatim}
  \item Floating-point exceptions may indicate poor geometry or convergence failure.
  \item For cluster usage, prepare a \texttt{.pbs} file:
\end{itemize}

\subsection*{Example PBS Script}

\begin{verbatim}
#!/bin/bash
#PBS -N turbomole_job
#PBS -l nodes=1:ppn=8
cd $PBS_O_WORKDIR
source /opt/turbomole/Config_turbo_env
jobex > job.out
\end{verbatim}
